\documentclass[a4paper,12pt]{article}

\usepackage[ngerman]{babel}
\usepackage[utf8]{inputenc}
\usepackage[T1]{fontenc}

\usepackage{epsfig}
\renewcommand{\baselinestretch}{1.0}
\usepackage{latexsym}
\usepackage{amsmath}
\usepackage{amssymb}

% PDF hyperlinks, e.g. click-able references like TOC lines and 'see table x on page y'
\usepackage{hyperref}
\hypersetup{colorlinks,
    citecolor=black,
    filecolor=black,
    linkcolor=black,
    urlcolor=black,
    plainpages=false}


\begin{document}

\title{Ausarbeitung zur Implementierungsaufgabe 3 -- AMBI 2021}
\author
{Magdalena Weber und Martin Brand}
\date{\today}
\maketitle

\section*{Zu Aufgabe 3}
Die Parameter für den Viterbi-Algorithmus wurden beim 2. Durchlauf so geändert, dass in der stochastischen Matrix die ersten 4 Zeilen nun den letzten 4 Zeilen entsprechen und umgekehrt. Man sieht, dass die Fehlerrate signifikant schlechter ist, wenn man den falschen Viterbi-Algorithmus benutzt.


\end{document}
